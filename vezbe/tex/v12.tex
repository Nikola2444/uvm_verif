%%%%%%%%%%%%%%%%%%%%%%%%%%%%%%%%%%%%%%%%%
%
% Funkcionalna verifikacija hardvera
% 
%%%%%%%%%%%%%%%%%%%%%%%%%%%%%%%%%%%%%%%%%

Vežba 12 je posvećena regresiji i \emph{debug}-ovanju kako verifikacionog
okruženja, tako i samog DUT-a. Opisane su funkcionalnosti UVM-a koje mogu
olakšati debug i ubrzati proces verifikacije, dat je opis naprednih mogućnosti
alata u pogledu sakupljanja pokrivenosti i spajanja tih podataka. Drugi deo
vežbe je posvećen greškama prilikom razvoja okruženja koje se često prave
(\emph{gotchas}), bilo zbog specifičnosti jezika ili zbog nerazumevanja
implementacije.

%========================================================================================
% Section
%========================================================================================

\section{\emph{Debug} verifikacionog okruženja}

Iako simulatori pružaju dosta koristih funkcionalnosti u cilju \emph{debug}-a,
postoji i dosta osobina u UVM bibilioteci koje olakšavaju pronalaženje i
ispravku problema u verifikacionom okruženju. U ovom poglavlju su opisane te
osobine.

% ----------------------------------------------------------------------------------------

\subsection{UVM \emph{Command Line Processor}}

Sa UVM procesorom komandne linije smo se već sreli na prethodnim vežbama. Ispod
su navedene korisne opcije koje mogu olakšati i skratiti vreme pronalaženja
problema.

\begin{itemize}

\item +UVM\(\_\)CONFIG\(\_\)DB\(\_\)TRACE - omogućava praćenje rada sa
  \emph{uvm\(\_\)config\(\_\)db} klasom. Kada je uključen, na izlazu će se
  ispisivati UVM\(\_\)INFO poruke koje daju informacije o tome kada su
  postavljeni ili preuzeti podaci. Npr:
  \begin{lstlisting}
UVM_INFO ../uvm-1.1a/src/base/uvm_resource_db.svh(129) @ 0 ns: reporter [CFGDB/SET] Configuration 'uvm_test_top.mem_interface' (type virtual mem_interface) set by  = (virtual mem_interface) ?
UVM_INFO ../uvm-1.1a/src/base/uvm_resource_db.svh(129) @ 0 ns: reporter [CFGDB/GET] Configuration 'uvm_test_top.mem_interface' (type virtual mem_interface) read by uvm_test_top = (virtual mem_interface) ?
  \end{lstlisting}

\item +UVM\(\_\)PHASE\(\_\)TRACE - olakšava praćenje UVM faza. Kada je uključena
  ispisuju se informacije o početku i završetku UVM faza. Npr:
  \begin{lstlisting}
# UVM_INFO ../uvm-1.1a/src/base/uvm_phase.svh(1114) @ 0 ns: reporter [PH/TRC/STRT] Phase 'uvm.uvm_sched.reset' (id=184) Starting phase
# UVM_INFO ../uvm-1.1a/src/base/uvm_phase.svh(1342) @ 80 ns: reporter [PH/TRC/DONE] Phase 'uvm.uvm_sched.reset' (id=184) Completed phase
  \end{lstlisting}

\item +UVM\(\_\)OBJECTION\(\_\)TRACE - na prethodnim vežbama je opisan mehanizam
  završetka testa uz podizanje/spuštanje prigovora. Ova opcija omogućava
  praćenje prigovora i ispisuje kada je neki prigovor podignut/spušten, kao i
  koliko ima aktivnih prigovora. Npr:
  \begin{lstlisting}
# UVM_INFO @ 0 ns: reset_objection [OBJTN_TRC] Object uvm_test_top raised 1 objection(s) (Raising Reset Objection): count=1  total=1
# UVM_INFO @ 0 ns: reset_objection [OBJTN_TRC] Object uvm_top added 1 objection(s) to its total (raised from source object uvm_test_top, Raising Reset Objection): count=0  total=1
# UVM_INFO @ 80 ns: reset_objection [OBJTN_TRC] Object uvm_test_top dropped 1 objection(s) (Dropping Reset Objection): count=0  total=0
  \end{lstlisting}

\item +UVM\(\_\)VERBOSITY - pruža mogućnost promene \emph{verbosity}-a u čitavom
  okruženju odnosno kontrolu ispisa poruka
  
\item +UVM\(\_\)MAX\(\_\)QUIT\(\_\)COUNT - pruža mogućnost završetka simulacije
  kada se dostigne dati broj grešaka u dizajnu. Podrazumevana vrednost je -1
  odnosno simulacija se nikad ne završava na osnovu broja grešaka. Pod brojem
  grešaka se podrazumeva broj prijavljenih \emph{uvm\(\_\)error}-a.
\end{itemize}

% ----------------------------------------------------------------------------------------

\subsection{Funkcije}

Pored zadavanja opcija preko komandne linije, u UVM-u postoji i veliki broj
funkcija koje se mogu koristiti za \emph{debug}. Neke od njih su:

\begin{itemize}
\item \emph{print\(\_\)topology} - funkcija za ispis topologije komponenti u
  verifikacionom okruženju. Prototip funkcije je: \emph{function void
    print\(\_\)topology (uvm\(\_\)printer printer = null)}, a uglavnom se poziva
  iz \emph{end\(\_\)of\(\_\)elaboration} faze.
  
\item \emph{debug\(\_\)connected\(\_\)to} - funkcija za debug TLM portova.
  Ispisuje mapu svih komponenti koje su povezane na dati port. Prototip funkcije
  je: \emph{function void debug\(\_\)connected\(\_\)to ( int level = 0, int
    max\(\_\)level = -1 )}, a uglavnom se poziva iz
  \emph{end\(\_\)of\(\_\)elaboration} faze.
  

\item \emph{debug\(\_\)provided\(\_\)to} - funkcija za debug TLM \emph{imp}-a.
  Ispisuje mapu svih komponenti koje su povezane na port za koji je povezan dati
  imp. Prototip funkcije je: \emph{function void debug\(\_\)provided\(\_\)to (
    int level = 0, int max\(\_\)level = -1 )}, a uglavnom se poziva iz
  \emph{end\(\_\)of\(\_\)elaboration} faze.
\end{itemize}

%========================================================================================
% Section
%========================================================================================

\section{Regresija}

Nakon verifikovanja osnovnih funkcionalnosti tj. nakon što osnovni testovi
prolaze bez grešaka, može se preći na sledeći korak u procesu verifikacije, a to
je upravo regresija. Regresija se vrši na svakom nivou hijerarhije kako bi se
uverili da nove osobine, ali i prethodno verifikovane osobine i dalje ispravno
funkcionišu. U praksi regresija znači puštanje svih (ili ispravno odabranih)
postojećih testova, sa nasumičnim \emph{seed}-ovima random generatora. Puštanje
testova sa nasumičnim seed-om ima za zadatak da postigne željenu funkcionalnu i
strukturnu pokrivenost i otkrije greške u dizajnu, ukoliko postoje. Zbog ovoga
je veoma bitno imati dobar model pokrivenosti koji omogućava praćenje stanja
verifikacije i eventualne modifikacije prilikom regresije. Nakon svake
pronađene greške i bilo kakve promene bilo dizajna bilo verifikacionog
okruženja, potrebno je ponovo pustiti regresiju kako bi se uverili da te promene
nisu izazvale nove greške.\\

Međutim, zbog kompleksnosti dizajna i okruženja regresija može biti veoma
zahtevna po pitanju vremena i resursa. Efikasnost regresije se može povećati na
dva načina - izborom optimalnih testova i puštanjem više testova u paraleli.
Prilikom izbora testova treba voditi računa o vremenu trajanja svakog testa, ali
i da li je moguće postići željenu pokrivenost sa manjim brojem testova. Ukoliko
je moguće, treba odabrati najmanji broj testova koji i dalje zadovoljava
pokrivenost.\\

Analizom izveštaja o pokrivenosti može se zaključiti da li je potrebno
modifikovati testove - uvoditi dodatna ograničenja randomizacije ili pisati
direktne testove. Cilj ovih modifikacija je pokriti ``rupe'' i dostići željenu
pokrivenost.

% ----------------------------------------------------------------------------------------

\subsection{Puštanje testova}

Glavni princip regresije je puštanje testova sa nasumično izabranim vrednostima
\emph{seed}-a. Zbog broja testova i velike količine podataka koje treba
proanalizirati, važno je dobro organizovati proces regresije. Zbog velikog broja
često nije potrebno imati detaljan log fajl za svaki test. Bitne informacije su
koja je bila vrednost \emph{seed}-a random generatora i da li je test prijavio
greške. Stim u vezi, postoji par podešavanja koja se često koriste prilikom
regresije:

\begin{itemize}
\item UVM\(\_\)VERBOSITY na nisku vrednost - fajlovi samo analiziraju u potrazi
  za greškama. Ukoliko je test javio grešku, pustiće se ponovo sa istim
  parametrima i tada detaljnije proanalizirati
\item UVM\(\_\)MAX\(\_\)QUIT\(\_\)COUNT - ukoliko test prijavi određeni broj
  grešaka često nije potrebno puštati test da se završi do kraja i time trošiti
  vreme
\end{itemize}

Kako bi se omogućilo puštanje testova sa nasumičnim vrednostima \emph{seed}-a,
potrebno je proslediti tu informaciju prilikom pokretanja simulacije. U
Incisive enterprise similator alatu se ovo radi na sledeći način:

\begin{lstlisting}
irun top_module.sv -svseed random
\end{lstlisting}

Pošto je potrebno puštati velik broj testova, nije nužno imati otvoren GUI već
se u praksi praktikuje puštanje simulacija iz komandne linije, odnosno prilikom
pokretanja irun komande neophodno je izostaviti -gui parametar. 


%========================================================================================
% Section
%========================================================================================

\section{Česte greške}

U ovom poglavlju su opisane česte greške koje se prave prilikom razvoja
verifikacionih okruženja. Tabela ispod sadrži opis problema i primere.

\begin{center}
  \begin{longtable}{|>{\columncolor{gray!30}}p{.07\textwidth} | p{.4\textwidth} | p{.4\textwidth} |}
    \hline
    % *************************
    \rowcolor{gray!30}
    \multicolumn{3}{|l|}{Upotreba \emph{signed} tipova podataka}\\
    \hline
    Opis & \multicolumn{2}{|p{.90\textwidth}|}{Kako bi se izbeglo korišćenje
      opširnih deklaracija promenljivih, mogu se koristiti neki predefinisani
      tipovi}\\
    \hline
    Primer & \multicolumn{2}{|p{.90\textwidth}|}{byte umesto bit [7:0] \newline
      byte je \emph{signed} tip čiji je opseg od -128 do +127, a ne od 0 do
      255}\\
    \hline
    % *************************
    \rowcolor{gray!30}
    \multicolumn{3}{|l|}{Deklaracija nizova}\\
    \hline
    Opis & \multicolumn{2}{|p{.90\textwidth}|}{Deklarisanje nizova koristeći samo
      jednu vrednost, a ne opseg [high:low]}\\
    \hline
    Primer & \multicolumn{2}{|p{.90\textwidth}|}{deklaracija niza \emph{x} od 256
      elemenata \newline x [256] je ekvivalentno sa x[0 : 255], a ne sa x[255 :
      0]}\\
    \hline
    % *************************
    \rowcolor{gray!30}
    \multicolumn{3}{|l|}{Podrazumevani argumenti u tasku / funkciji}\\
    \hline
    Opis & \multicolumn{2}{|p{.90\textwidth}|}{Podrazumevani tip argumenta je isti
      kao i prethodno naveden tip}\\
    \hline
    Primer & \multicolumn{2}{|p{.90\textwidth}|}{task example (ref int array[7],
      int a, b);  // a i b su ref \newline task example (ref int array[7], input
      int a, b);  // eksplicitno navesti da su a i b input}\\
    \hline
    % *************************
    \rowcolor{gray!30}
    \multicolumn{3}{|l|}{Inicijalizacija promenljivih}\\
    \hline
    Opis & \multicolumn{2}{|p{.90\textwidth}|}{Problem se može javiti ukoliko se
      lokalna promenljiva inicijalizuje prilikom deklaracije jer u tom slučučaju
      promenljiva inicijalizuje prilikom starta simulacije}\\
    \hline
    Primer & Problem: \newline int x = 5; & Rešenje: \newline int x; \newline x
    = 5;\\
    \hline
    % *************************
    \rowcolor{gray!30}
    \multicolumn{3}{|l|}{Više operatora u izrazu}\\
    \hline
    Opis & \multicolumn{2}{|p{.90\textwidth}|}{U izrazu treba da postoji maksimalno
      jedan operator npr. \textless , \(\leq\), ==, \(\geq\), ili \textgreater \
      jer se izraz podeli u više binarnih izraza koji se evaluiraju s leva na
      desno}\\
    \hline
    Primer & Problem: & Rešenje:\\
    &\begin{lstlisting}
class Example;
  rand bit [7 : 0] lo, med, hi;
  constraint ex_con {lo < med < hi;}
endclass
\end{lstlisting}&
    \begin{lstlisting}
class Example;
  rand bit [7 : 0] lo, med, hi;
  constraint ex_con {lo < med; med < hi;}
endclass
\end{lstlisting}\\
    & Primer rezultata prethodnog koda: lo = 20, med = 244, hi = 164 (izraz lo
    \textless  med \textless  hi postaje ((lo \textless  med) \textless  high);
    prvo se evaluira lo \textless  med što daje rezultat 0 ili 1 i ta vrednost
    se poredi sa hi) & \\
    \hline
    % *************************
    \rowcolor{gray!30}
    \multicolumn{3}{|l|}{Korišćenje događaja u petlji}\\
    \hline
    Opis & \multicolumn{2}{|p{.90\textwidth}|}{Ukoliko se u petlji čeka na
      \emph{level-sensitive} događaj, potrebno je preći u sledeće vreme kako bi
      se izbegla \emph{zero-delay} petlja. \emph{Wait} blokira jednom u
      simulacionom trenutku.}\\
    \hline
    Primer & Problem: & Rešenje:\\
    &\begin{lstlisting}
forever begin
  wait(e1.triggered);
  process_in_zero_time();
end
\end{lstlisting}&
    \begin{lstlisting}
forever begin
  @(e1);
  process_in_zero_time();
end
\end{lstlisting}\\
    \hline
    % *************************
    \rowcolor{gray!30}
    \multicolumn{3}{|l|}{Razlika između ``or'' i ``\(\|\)''}\\
    \hline
    Opis & \multicolumn{2}{|p{.90\textwidth}|}{``or'' je separator, a ne operacija,
      za razliku od ``\(\|\)'' što je logička operacija ILI}\\
    \hline
    Primer & Problem: & Rešenje:\\
    &\begin{lstlisting}
always @(a || b) ...
\end{lstlisting}&
    \begin{lstlisting}
always @(a or b) ...
\end{lstlisting}\\
    & Lista je osetljiva na rezultat \emph{a} ili \emph{b}, a ne na bilo koju
    promenu signala \emph{a} i \emph{b}. Npr. ukoliko je \emph{a=1}, a \emph{b}
    promeni vrednost sa 0 na 1 blok se neće trigerovati & Lista je osetljiva na
    promene signala \emph{a} i \emph{b}. Trigerovaće se na bilo koju promenu
    signala \emph{a} i \emph{b}, bez obzira na vrednost\\
    \hline
    % *************************
    \rowcolor{gray!30}
    \multicolumn{3}{|l|}{\emph{Handle}, a ne objekat}\\
    \hline
    Opis & \multicolumn{2}{|p{.90\textwidth}|}{Važno je uočiti razliku između
      \emph{handle}-a (pokazivača) i samog objekta. Pokazivač se deklariše, a
      objekat konstruiše. Jedan pokazivač može pokazivati na više objekata u
      toku simulacije.}\\
    \hline
    Primer & \multicolumn{2}{|}{}
    \begin{minipage}{.90\textwidth}
      \begin{lstlisting}
 Transaction t1, t2; // deklaracija dva pokazivaca
 t1 = new();         // alociranje prvog objekta tipa Transaction
 t2 = t1;            // i t1 i t2 pokazuju na isti objekat
 t1 = new();         // alociranje drugog objekta tipa Transaction
\end{lstlisting}
    \end{minipage}\hfill\vline\kern-\arrayrulewidth \\
    \hline
    % *************************
    \rowcolor{gray!30}
    \multicolumn{3}{|l|}{Granice u \emph{for} petlji}\\
    \hline
    Opis & \multicolumn{2}{|p{.90\textwidth}|}{\emph{For} petlje se izvšavaju do
      god iterator ne dostigne zadatu vrednost, ali je moguće zadati vrednost
      van opsega iteratora}\\
    \hline
    Primer & \multicolumn{2}{|}{}
    \begin{minipage}{.90\textwidth}
      \begin{lstlisting}
 bit [3 : 0] i;
 for (i = 0; i < 50; i++) begin
   // ... 
 end
\end{lstlisting}
    \end{minipage}\hfill\vline\kern-\arrayrulewidth \\
    & \multicolumn{2}{p{.90\textwidth}}{Simulator neće javiti grešku. Najsigurnije
      je ipak koristiti int za tip kod iteratora petlje.}\\
    \hline
    % *************************
    \rowcolor{gray!30}
    \multicolumn{3}{|l|}{Provera uspešnosti randomizacije}\\
    \hline
    Opis & \multicolumn{2}{|p{.90\textwidth}|}{Zbog velikog broja ograničenja u
      verifikacionom okruženju i mogućnosti dodavanja novih ograničenja tokom
      rada, moguće je da randomizacija neće uspeti}\\
    \hline
    Primer & \multicolumn{2}{|}{}
    \begin{minipage}{.90\textwidth}
      \begin{lstlisting}
 assert(x.randomize());
\end{lstlisting}
    \end{minipage}\hfill\vline\kern-\arrayrulewidth \\
    \hline
  \end{longtable}
\end{center}

%========================================================================================
% Section
%========================================================================================

\section{Zadaci}

\paragraph{Zadatak}

U pretećim materijalima za vežbu se nalazi fajl ``v12\(\_\)gotchas\(\_\)examples.sv'' u
kome su prikazane često pravljene greške. Uočiti i ispraviti sve greške u kodu
kako bi se u simulaciji ispravno ispisivalo 10 nasumičnih transakcija (nasumično
odabran vrednosti polja \emph{addr} i \emph{data}) prilikom svake promene
vrednosti polja \emph{e1} i \emph{e2}, pri čemu \emph{data} uvek ima vrednost
između 5 i 10, a \emph{addr} 2 ili 3.

\paragraph{Zadatak}

Pustiti regresiju za primer ``Calc1'' dizajna uz kreirano verifikaciono
okruženje. Analizirati rezultate. Generisati podatke o pokrivenosti.

%========================================================================================
% Section
%========================================================================================

\section{Appendix}

\lstinputlisting[caption=v12\(\_\)gotchas\(\_\)examples, label=lst:v12_gotchas_examples]{code/v12_gotchas_examples.sv}







