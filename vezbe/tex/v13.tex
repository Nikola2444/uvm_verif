%%%%%%%%%%%%%%%%%%%%%%%%%%%%%%%%%%%%%%%%%
%
% Funkcionalna verifikacija hardvera
% 
%%%%%%%%%%%%%%%%%%%%%%%%%%%%%%%%%%%%%%%%%

Vežba 13 je posvećena ponovnoj upotrebi koda. Dati su primeri univerzalnih
verifikacionih komponenti za dva protokola, kao i reset agent. Ove komponente se
mogu integrisati u bilo koje okruženje i na jednostavan način prilagoditi
trenutnim potrebama.\\

Za detaljnije objašnjene pogledati fajlove unutar docs foldera odgovarajuće
komponente.

%========================================================================================
% Section
%========================================================================================

\section{Zadaci}

\paragraph{Zadatak}

U pratećim materijalima za ovu vežbu, nalazi se univerzalna verifikaciona
komponenta za APB protokol. Ova komponenta omogućava veoma laku ponovnu upotrebu
koda jer se brzo može integrisati u bilo koje okruženje i prilagoditi datim
potrebama. Proučiti date fajlove i uočiti strukturu UVC-a. Šta sve sadrži UVC?
Kako se može podešavati? Pokrenuti dati primer i analizirati rezultate (skripta
\emph{run.do} unutar \emph{sim} foldera pokreće primer dat u \emph{examples}
folderu).

\paragraph{Zadatak}

Napisati nove testove za APB koji omogućavaju:

\begin{itemize}
\item čitanje sa svih validnih adresa
\item upis podatka 0 na 25 nasumičnih adresa
\item uzastopan upis i čitanje sa istih adresa, pri čemu su upisani podaci
  nasumični
\end{itemize}

\paragraph{Zadatak}

Modifikovati APB test tako da se koriste dva \emph{slave} agenta. Obratiti
pažnju na opseg adresa i vrednost \emph{psel} indeksa.

\paragraph{Zadatak}

Kreirati \emph{scoreboard} koji vrši proveru ispravnosti obzerviranih
transakcija odnosno poredi podatke primljene od mastera i \emph{slave}-ova
(\emph{scoreboard} povezati sa odgovarajućim monitorima uz TLM).

\paragraph{Zadatak}

U pratećim materijalima je takođe dat i reset agent. Ova komponenta služi za
generisanje reset signala nasumičnog trajanja. Proučiti date fajlove. Koja
podešavanja su moguća u agentu?

\paragraph{Zadatak}

Modifikovati APB bazni test i top modul tako da se koristi reset agent.
Generisati reset signal u nasumičnom trenutku (pokrenuti odgovarajuću sekvencu).

\paragraph{Zadatak}

U pratećim materijalima za ovu vežbu, nalazi se univerzalna verifikaciona
komponenta za I2C protokol. Ova komponenta omogućava veoma laku ponovnu upotrebu
koda jer se brzo može integrisati u bilo koje okruženje i prilagoditi datim
potrebama. Proučiti date fajlove i uočiti strukturu UVC-a. Šta sve sadrži UVC?
Kako se može podešavati? Pokrenuti dati primer i analizirati rezultate (skripta
\emph{run.do} unutar \emph{sim} foldera pokreće primer dat u \emph{examples}
folderu).

\paragraph{Zadatak}

Napisati nove testove za I2C koji omogućavaju:

\begin{itemize}
\item master/slave agenti nikad ne odgovaraju sa NACK
\item čitanje sa svih adresa
\item upis podatka ‘0 na 25 nasumičnih adresa
\item uzastopan upis i čitanje sa istih adresa, pri čemu su upisani podaci
  nasumični
\end{itemize}

\paragraph{Zadatak}

Kreirati \emph{scoreboard} koji vrši proveru ispravnosti obzerviranih
transakcija odnosno poredi podatke primljene od mastera i \emph{slave}-ova
(\emph{scoreboard} povezati sa odgovarajućim monitorima uz TLM).

\paragraph{Zadatak}

Modifikovati I2C bazni test i top modul tako da se koristi reset agent.
Generisati reset signal u nasumičnom trenutku (pokrenuti odgovarajuću sekvencu).

%========================================================================================

