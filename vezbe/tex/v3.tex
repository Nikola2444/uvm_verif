%%%%%%%%%%%%%%%%%%%%%%%%%%%%%%%%%%%%%%%%%
%
% Funkcionalna verifikacija hardvera
% 
%%%%%%%%%%%%%%%%%%%%%%%%%%%%%%%%%%%%%%%%%

Treća vežba je posvećena korišćenju niti (engl. \emph{threads}) u SystemVerilog
jeziku, kao i komunikaciji između njih. Pored opisa samih niti i različitih
načina korišćenja, opisani su događaji (engl. \emph{events}), semafori i
\emph{mailbox} koji omugućuju komunikaciju i sinhronizaciju između njih.\\

Napomena: iako postoji dosta razlika između niti (\emph{threads}) i procesa, u
većini literature o SystemVerilog-u se ova dva termina pođednako koriste
(prvenstveno zbog naizmeničnog korišćenja u SystemVerilog LRM-u). Pa se i
događaji i semafori često nazivaju objektima za inter-procesnu komunikaciju.

%========================================================================================
% Section
%========================================================================================

\section{Niti}

U ovom poglavlju je dat pregled niti. Opisani su različini načini kreiranja,
sinhronizacija i dati primeri upotrebe.

%----------------------------------------------------------------------------------------

\subsection{Uvod}

Testbenč je često komplesna struktura, sačinjena od više komponenti koje moraju
raditi u paraleli (npr. drajver će generisati podatke na interfejs, monitor će
ih prikupljati, \emph{scoreboard} proveravati, itd.). Kako bi se omogućio
nesmetan rad svih blokova, potrebno je omogućiti da se svaki zadatak izvršava
nezavisno od ostatka okruženja, a to se najlakše postiže korišćenjem niti. Niti
su nezavisne sekvence instrukcija koje se izvšavaju u paraleli. Olakšavaju
pisanje kompleksnih verifikacionih okruženja i čine kod čitljivijim i
efikasnijim.

%----------------------------------------------------------------------------------------

\subsection{\emph{Fork..join}}

Niti se kreiraju koristeći \emph{fork..join} konstrukciju. Svaka naredba unutar
ove konstrukcije predstavlja zasebnu nit koja će se izvršavati u paraleli sa
ostatkom naredbi. U nastavku je dat primer upotrebe i rezultat izvršavanja datog
koda.

\lstinputlisting[caption=fork..join, label=lst:fork_join]{code/v3_fork_join.sv}

Prethodni kodni fragment će kreirati tri niti. Primetiti da \emph{begin..end}
blok i sve naredbe koje su u njemu objedinjene kreiraju jednu nit. Takođe,
primetiti da se sa nastavkom izvršavanja koda (nakon \emph{join} naredbe) kreće
tek nakon što se završe sve niti unutar \emph{fork..join} konstrukcije.

%----------------------------------------------------------------------------------------

\subsection{\emph{join\(\_\)any, join\(\_\)none}}

Ponekad nije zgodno čekati da se sve niti završe kako bi se nastavilo sa
ostatkom koda. Možda se neke niti vrte u beskonačnoj petlji, možda neke čekaju
na događaje koji se nikada ne dese i sl. Zbog toga se, osim \emph{fork..join}
konstrukcije obezbeđuju još dva načina manipulacije koda - koristeći
\emph{fork..join\(\_\)any} i \emph{fork..join\(\_\)none}.\\

Kao i \emph{fork..join}, niti će se kreirati na potpuno isti način. Jedina
razlika je u načinu nastavka izvršavanja koda nakon \emph{join} naredbe.
\emph{Join\(\_\)any} čeka da se završi bar jedna nit, dok \emph{join\(\_\)none}
uopšte ne čeka na niti i odmah kreće sa izvršavanjem ostalih naredbi. Primeri su
dati ispod.

\lstinputlisting[caption=join any, label=lst:fork_join_any]{code/v3_fork_join_any.sv}

\lstinputlisting[caption=join none, label=lst:fork_join_none]{code/v3_fork_join_none.sv}

%----------------------------------------------------------------------------------------

\subsection{Kontrola}

Kontrola nad kreiranim nitima se može vršiti čekanjem na nit ili uništavanjem
niti koristeći \emph{wait fork} i \emph{disable fork} naredbe.

\subsubsection{\emph{Wait fork}}

\emph{Wait fork} naredba se koristi kako bi bili sigurni da su se sve
\emph{child} niti završile (sve niti kreirane od strane niti koja poziva
\emph{wait fork}). U primeru ispod, u \emph{initial bloku} se prvo kreira jedna
nit koristeći \emph{fork..join\(\_\)none} konstrukciju, a zatim se kreiraju još
dve niti u \emph{fork..join\(\_\)any} bloku. Ove tri niti predstavljaju
\emph{child} niti \emph{initial} bloka i \emph{wait fork} naredba čeka na
završetak sve tri niti pre nego što se nastavi izvršavanje ostalih naredbi.

\lstinputlisting[caption=wait fork, label=lst:wait_fork]{code/v3_wait_fork.sv}

\subsubsection{\emph{Disable fork}}

\emph{Disable fork} naredba terminira sve \emph{child} niti. Ukoliko te
\emph{child} niti imaju svoje \emph{child} niti i one će biti terminirane.
Primetiti da se u datom primeru naredbe koje ispisuju ``1st example'' i ``4th
example'' neće izvršiti jer su niti terminirane pre nego što su stigle do ispisa
ovih poruka.

\lstinputlisting[caption=disable fork, label=lst:disable_fork]{code/v3_disable_fork.sv}

%========================================================================================
% Section
%========================================================================================

\section{Inter-procesna komunikacija}

Zbog velikog broja niti u većini verifikacionih okruženja i česte potrebe za
međusobnom komunikacijom i sinhronizacijom, potrebno je obezbediti bezbezbedan
i jednostavan način za pisanje \emph{thread-safe} okruženja. Ovo se postiže
korišćenjem događaja (\emph{events}), semafora i \emph{mailbox}-ova.

%----------------------------------------------------------------------------------------

\subsection{Događaji}

SystemVerilog podržava \emph{event} tip koji olakšava sinhronizaciju između
niti, uz pomoć operatora ``@'' i ``\(\rightarrow\)'' za čekanje na događaj i
trigerovanje događaja. Događaji se deklarišu koristeći ključnu reč
\emph{event} praćenu sa imenom događaja. U sledećem primeru se u prvom
\emph{initial} bloku trigeruje \emph{event} e1, na koji se čeka u drugom
\emph{initial} bloku.

\lstinputlisting[caption=Primer događaja, label=lst:simple_event]{code/v3_simple_event.sv}

``@'' operator je \emph{edge-sensitive}, što može prouzrokovati probleme i
neželjene situacije ukoliko se ne koristi pravilno. Zbog toga SystemVerilog
omogućuje i korišćenje \emph{level-sensitive} operatora \emph{wait} i
\emph{triggered} osobine. U narednom primeru se vidi razlika između korišćenja
ova dva operatora:

\lstinputlisting[caption=Događaji osetljivi na nivo i ivicu, label=lst:event_sensitivity]{code/v3_event_sensitivity.sv}

U prvom primeru se druga nit nikad neće završiti pošto nije uhvatila
\emph{zero-delay} događaj. Dok će se u drugom primeru izvršiti zato što
\emph{level-sensitive} naredba hvata sve događaje koji su se izvršili u tom
simulacionom vremenu.

%----------------------------------------------------------------------------------------

\subsection{Semafori}

Semafor je sinhronizacioni objekat koji omogućava ekskluzivan pristup deljenom
objektu. Npr. ukoliko imamo jednu promenljivu kojoj se pristupa iz više niti,
može doći do neočekivanih rezultata ukoliko, u isto vreme, jedna nit pokušava da
pročita vrednost promenljive, dok druga nit vrši upis. Semafori obezbeđuju
\emph{lock} mehanizam pristupa. Nit pre pristupa proverava da li je promenljiva
``zaključana'', ukoliko nije, uzima ključ i vrši pristup toj promenljivoj. Do
god nit ne završi sa radom i ne vrati ključ, nijedna druga nit ne može
pristupati toj promenljivoj. Semafor može sadržati i više od jednog ključa. Do
god ima slobodnih ključeva niti ih mogu uzimati i nastavljati svoj rad, a
ukoliko ih nema, moraju čekati da se ključevi oslobode.\\

Postoji nekoliko metoda za rad sa semaforima:

\begin{itemize}
\item \emph{new(num\(\_\)of\(\_\)keys)} - kreira semafor sa željenim brojem
  ključeva. Podrazumevana vrednost za \emph{num\(\_\)of\(\_\)keys} je 0.
\item \emph{get(num\(\_\)of\(\_\)keys)} - traži \emph{num\(\_\)of\(\_\)keys}
  ključeva. Ukoliko su dostupni, uzima ključeve i nastavlja se sa radom, a
  ukoliko nisu, tred blokira dok ne postanu dostupni. Podrazumevana vrednost za
  \emph{num\(\_\)of\(\_\)keys} je 1.
\item \emph{put(num\(\_\)of\(\_\)keys)} – vraća prethodno uzete ključeve u
  semafor. Podrazumevana vrednost za \emph{num\(\_\)of\(\_\)keys} je 1.
\item \emph{try\(\_\)get(num\(\_\)of\(\_\)keys)} – slično kao \emph{get()}, ali
  ne blokira ukoliko ključevi nisu dostupni već samo vraća 0 da signalizira ovu
  situaciju. Podrazumevana vrednost za \emph{num\(\_\)of\(\_\)keys} je 1.
\end{itemize}

Treba obratiti paznju pri radu sa više ključeva. Moguće je vratiti više ključeva
nego što je uzeto, što može prouzrokovati probleme u daljem radu.

\lstinputlisting[caption=Semafori, label=lst:semaphores]{code/v3_semaphores.sv}

%----------------------------------------------------------------------------------------

\subsection{\emph{Mailbox}}

\emph{Mailbox} je mehanizam koji omogućava slanje poruka iz jednog procesa ka
drugom.
Ponašanje \emph{mailbox}-ova podseća na poštansko sanduče od kojeg i dobija ime.
Pismo (podatak) se sprema i ostavlja u sanduče odakle se kasnije preuzima.
Ukoliko prilikom provere sandučeta nijedno pismo nije stiglo može se odabrati
jedna od dve akcije: sačekati da stigne pismo ili proveriti kasnije.
Takođe je moguće odabrati između dva tipa sandučeta: sa neograničenim ili
ograničenim kapacitetom.
Ukoliko se sanduče sa ograničenim kapacitetom napuni, proces koji šalje podatak
će čekati dok se podaci ne preuzmu i oslobodi prostor u sandučetu.\\

U SystemVerilog-u postoji veliki broj metoda za rad sa \emph{mailbox}-ovima.
Njihovi prototipi su:
\begin{itemize}
\item function new(int bound = 0) - konstruktor. Kreira nov \emph{mailbox}. Kao
  argument prima kapacitet sandučeta. Podrazumevana vrednost je 0 odnosno
  neograničeno sanduče.
\item function int num() - vraća broj poruka u sandučetu.
\item task put(podatak) - staviti podatak u \emph{mailbox}. Ukoliko je sanduče puno,
  metoda će blokirati do god ne oslobodi prostor i završi slanje.
\item function int try\(\_\)put(podatak) - kao \emph{put}, ali bez
  blokiranja. Funkcija vraća \emph{int} koji signalizira da li je slanje
  uspešno.
\item task get(ref poruka) - blokirajuća metoda za preuzimanje poruke.
\item function int try\(\_\)get(ref poruka) - neblokirajuća metoda za
  preuzimanje poruke.
\item task peek(ref poruka) - blokirajuća metoda koja kopira poruku, ali je ne
  preuzima (poruka ostaje u sandučetu).
\item function int try\(\_\)peek(ref poruka) - neblokirajuća verzija \emph{peek}
  metode.
\end{itemize}

\emph{Mailbox} je podrazumevano netipiziran odnosno može da prima bilo koji tip
podatka.
Iako veoma korisna, ova osobina često uvodi greške prilikom pokušaja čitanja
pogrešnog tipa (npr. ukoliko se tip podatka u \emph{get} metodi ne poklapa sa
podatkom dostupnim u sandučetu).
Kako bi se izbegle ove greške, \emph{mailbox}-ove je moguće parametrizovati.\\

U nastavku je dato nekoliko primera upotrebe \emph{mailbox}-ova.

\lstinputlisting[caption=Mailbox, label=lst:mailbox]{code/v3_mailbox.sv}

%========================================================================================
% Section
%========================================================================================

\section{Zadaci}

\paragraph{Zadatak}

Kod \ref{lst:fork_examples} sadrži primere upotrebe niti (fajl
``fork\(\_\)examples.sv'' u pratećim materijalima).
Analizirati konstrukcije.
Koji će biti rezultat izvršavanja svakog primera?
Zašto?

\lstinputlisting[caption=Primer, label=lst:fork_examples]{code/v3_fork_examples.sv}

\paragraph{Zadatak}

Za primer testbenča sa prošle vežbe, modifikovati drajver i monitor klase tako
da budu osetljivi na reset signal (u obe klase neka se ispiše da je reset uočen
i drajver ne sme slati transakcije do god je reset aktivan). Nasumično
generisati reset signal iz top modula i proveriti da li komponente ispravno
reaguju.

%========================================================================================
