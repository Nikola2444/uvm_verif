\documentclass{article}

\usepackage{hyperref}
\usepackage[utf8]{inputenc} % Unicode support (Umlauts etc.)

% po srpski :)
\renewcommand{\figurename}{Slika}
\renewcommand{\tablename}{Tabela}
\renewcommand{\contentsname}{Sadržaj}

\title{
  \textmd{\textbf{APB VIP}}
}
\author{}
\date{}


\begin{document}

\maketitle

APB VIP je univerzalna UVM komponenta koja implementira APB master i slave
agente prateći UVM 1.2 metodologiju.

Specifikacija APB protokola se može pronaći ovde:
\url{http://web.eecs.umich.edu/~prabal/teaching/eecs373-f12/readings/ARM_AMBA3_APB.pdf}

\section{Upotreba}

\begin{itemize}
\item Instancirati \emph{apb\(\_\)env} (po uzoru na
  examples/tests/apb\(\_\)test\(\_\)base.sv)
\item Instancirati, podesiti odgovarajuća polja i proslediti konfiguracioni
  objekat pomoću uvm\(\_\)config\(\_\)db. Ukoliko se preskoči ovaj korak,
  koristiće se podrazumevana konfiguracija (jedan master agent). Da bi se dodali
  master i slave agenti pozivati metode add\(\_\)master i add\(\_\)slave,
  respektivno.
  Napomena: u slučaju slave agenta postoji i dodatno polje create\(\_\)agent
  koje govori da li je potrebno kreirati agent ili će dati slave biti zapravo
  DUT. Ovo znači da je i u slučaju korišćenja samo master agenta potrebno
  pozvati ovu metodu i podesiti odgovarajuće adrese i indeks. Ovaj korak je
  neophodan kako bi master mogao da ispravno generise PSEL indeks na osnovu
  zadate adrese.
\item Importovati apb\(\_\)pkg u top modulu i povezati i proslediti interfejs
  (po uzoru na ./examples/apb\(\_\)test\(\_\)top.sv)
\item Ukoliko je potrebno, povezati TLM port odgovarajućih monitora sa željenom
  komponentom (hijerarhijska putanje su:
  <dato\(\_\)ime\(\_\)env>.slaves[i].mon.item\(\_\)collected\(\_\)port i
  <dato\(\_\)ime\(\_\)env>.master.mon.item\(\_\)collected\(\_\)port)
\item Pokrenuti odgovarajuće sekvence (po uzoru na
  ./examples/tests/apb\(\_\)test\(\_\)simple.sv)
\item Kopajlirati i pokrenuti simulaciju po uzoru na date skripte (./sim/run\(\_\)script.do)
\end{itemize}

\end{document}