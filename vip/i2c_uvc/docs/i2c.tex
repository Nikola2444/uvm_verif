\documentclass{article}

\usepackage{hyperref}
\usepackage[utf8]{inputenc} % Unicode support (Umlauts etc.)

% po srpski :)
\renewcommand{\figurename}{Slika}
\renewcommand{\tablename}{Tabela}
\renewcommand{\contentsname}{Sadržaj}

\title{
  \textmd{\textbf{I2C VIP}}
}
\author{}
\date{}


\begin{document}

\maketitle

I2C VIP je univerzalna UVM komponenta koja implementira I2C master i slave
agente prateći UVM 1.2 metodologiju.

Specifikacija I2C protokola se može pronaći ovde:
\url{https://www.nxp.com/docs/en/user-guide/UM10204.pdf}

\section{Upotreba}

\begin{itemize}
\item Instancirati \emph{i2c\(\_\)env} (po uzoru na
  examples/tests/i2c\(\_\)test\(\_\)base.sv)
\item Instancirati, podesiti odgovarajuća polja i proslediti konfiguracioni
  objekat pomoću uvm\(\_\)config\(\_\)db. Ukoliko se preskoči ovaj korak,
  koristiće se podrazumevana konfiguracija (jedan master agent). Da bi se dodali
  master i slave agenti pozivati metode add\(\_\)master i add\(\_\)slave,
  respektivno.
\item Importovati i2c\(\_\)pkg u top modulu i povezati i proslediti interfejs
  (po uzoru na ./examples/i2c\(\_\)test\(\_\)top.sv)
\item Ukoliko je potrebno, povezati TLM port odgovarajućih monitora sa željenom
  komponentom (hijerarhijska putanje su:
  <dato\(\_\)ime\(\_\)env>.slaves[i].mon.item\(\_\)collected\(\_\)port i
  <dato\(\_\)ime\(\_\)env>.master.mon.item\(\_\)collected\(\_\)port)
\item Pokrenuti odgovarajuće sekvence (po uzoru na
  ./examples/tests/i2c\(\_\)test\(\_\)simple.sv)
\item Kopajlirati i pokrenuti simulaciju po uzoru na date skripte (./sim/run\(\_\)script.do)
\end{itemize}

\section{Napomene}

\begin{itemize}
\item I2C UVC ne podržava rad sa više master i više slave agenata
\item I2C UVC ne podržava produženi start
\item I2C UVC ne podržava režim rada sa 10-obitnim adresama
\end{itemize}

\end{document}